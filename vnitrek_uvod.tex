Závěrečnou diplomovou práci ke studijnímu oboru Aplikace softwarového inženýrství s~ná\-zvem Porovnání účinnosti komprese dat ve formátu XML a JSON jsem si vybral z~důvodu aktuálnosti, XML a JSON jsou v současnosti jedny z nejpoužívanější textových datových formátů, a také proto, že toto téma velmi dobře propojuje teoretické znalosti získané při studiu s praktickými zkušenostmi v oboru softwarového inženýrství.

Dle výzkumu International Data Corporation (IDC) The Digital Universe of Opportunities: Rich Data and the Increasing Value of the Internet of Things \cite{idc} bylo pouze v roce 2014 vytvořeno a zkonzumovýno 2837~EB (exabytů) dat. Ze závěrů vyplývá, že se toto číslo každé dva roky zdvojnásobí, tedy v roce 2020 to bude již přibližně 40000 EB. Takové množství dat klade enormní požadavky na přenosové kanály a datová úložiště. Jako příklad mohou sloužit miliony shlédnutí oblíbených videí na serveru youtube.com. Pouze jedna sekunda videa v nekomprimovaném formátu CCIR 601 zabere více než 20 MB, tímto způsobem by zmiňovaná služba nemohla fungovat.

Vzhledem k tomu, že jsou technologie omezeny současnými možnostmi, znalostmi a také fyzikálními limity, je nutné hledat řešení jinde než v jejich zlepšování. Zde přichází na řadu komprese dat jako účinná metoda snížení velikosti objemu přenášených a ukládaných dat. Ve své práci bych čtenáře rád seznámil se základními principy komprese a vybranými kompresními algoritmy. Hlavním cílem je ale zodpovědět otázku, zda je možné dosáhnout dalších úspor volbou XML nebo JSON formátu a vhodného algoritmu, který využije znalosti struktury datového formátu.

%Práce se skládá celkem z X kapitol
%V první kapitole je čtenář seznámen
%Ve druhé kapitole představuji
%Třetí kapitola je věnována
%Získaná hovna využívám v kapitole 4