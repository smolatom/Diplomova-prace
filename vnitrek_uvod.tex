Závěrečnou diplomovou práci ke studijnímu oboru Aplikace softwarového inženýrství s~ná\-zvem Porovnání účinnosti komprese dat ve formátu XML a JSON jsem si vybral z~důvodu aktuálnosti, XML a JSON jsou v současnosti jedny z nejpoužívanější textových datových formátů, a také proto, že toto téma velmi dobře propojuje teoretické znalosti získané při studiu s praktickými zkušenostmi v oboru softwarového inženýrství.

Dle výzkumu International Data Corporation (IDC) The Digital Universe of Opportunities: Rich Data and the Increasing Value of the Internet of Things \cite{idc} bylo pouze v roce 2014 vytvořeno a zkonzumovýno 2837~EB (exabytů) dat. Ze závěrů vyplývá, že se toto číslo každé dva roky zdvojnásobí, tedy v roce 2020 to bude již přibližně 40000 EB. Takové množství dat klade enormní požadavky na přenosové kanály a datová úložiště. Jako příklad mohou sloužit miliony shlédnutí oblíbených videí na serveru youtube.com. Pouze jedna sekunda videa v nekomprimovaném formátu CCIR 601 zabere více než 20 MB, tímto způsobem by zmiňovaná služba nemohla fungovat.

Vzhledem k tomu, že jsou technologie omezeny současnými možnostmi, znalostmi a také fyzikálními limity, je nutné hledat řešení jinde než v jejich zlepšování. Zde přichází na řadu komprese dat jako účinná metoda snížení velikosti objemu přenášených a ukládaných dat. Ve své práci bych čtenáře rád seznámil se základními principy komprese a vybranými kompresními algoritmy. Hlavním cílem je ale zodpovědět otázku, zda je možné dosáhnout dalších úspor volbou XML nebo JSON formátu a vhodného algoritmu, který využije znalosti struktury datového formátu.

Tato práce soustřeďující se na využití znalosti struktury dat při kompresi se skládá celkem z 8 kapitol. Postupně jsou čtenáři představeny porovnávané datové formáty XML a~JSON, základní techniky komprese, existující kompresní algoritmy, mnou implementované algoritmy a na závěr porovnání vybraných algoritmů na testovacích datech.

První kapitola otevírá teoretickou část práce a popisuje porovnávané formáty a to jak z~pohledu důvodů vzniku a použitelnosti, tak i z pohledu syntaktické analýzy. Zápis dat pomocí obou formátů je znázorněn na krátkých příkladech. Tyto znalosti jsou stěžejní pro praktickou část práce.

Druhá, třetí a čtvrtá kapitola jsou věnovány základním pojmům kompresního procesu a~vybraným technikám bezeztrátové komprese. Pomocí příkladů jsou zde čtenáři názorně vysvětleny na příkladech způsoby přístupu ke kompresi známých algoritmů.

V kapitolách 5 a 6 jsou představeny algoritmy využívající znalosti struktury dat v XML a JSON. Důraz je kladen především na popis využití této znalosti a způsobu zpracování dat. Tímto je uzavřena teoretická část práce a získané poznatky jsou aplikovány v části praktické.

Praktická část se skládá ze dvou kapitol. V první popisuji způsob implementace vybraných kompresních algoritmů, který by měl vnést hlubší pohled do problematiky a odkrýt potenciální slabá místa. Poslední kapitola je věnována vzájemnému porovnání účinnosti komprese na data ve formátech XML a JSON. Hodnoty naměřené při testování jsou zobrazeny v přehledných grafech s vysvětlením příčin, které k tomuto výsledku mohly vést.

V závěru práce jsou na základě naměřených hodnot shrnuty a vyhodnoceny poznatky o účinnosti komprese využívající znalosti struktury datového formátu.