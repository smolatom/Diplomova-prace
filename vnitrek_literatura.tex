% !!! Literatura se má seřadit abecedně.
\begin{thebibliography}{99}
\bibitem{teorieKodovani} MAREŠ, Jan. {\em Teorie kódování.} 1. vyd. Praha: Česká technika, 2008. 120 s. ISBN 978-80-01-04203-8.
\bibitem{json} Standard ECMA-404. {\em The JSON Data Interchange Format.} Geneva: Ecma International, 2013, [online], [cit. 28. října 2014]. Dostupné na: {\tt <http://www.ecma-international.org/publications/files/ECMA-ST/ECMA-404.pdf>}.
\bibitem{idc} The Digital Universe of Opportunities: Rich Data and the Increasing Value of the Internet of Things. INTERNATIONAL DATA CORPORATION. {\em EMC} [online]. 2014 [cit. 2015-02-25]. Dostupné z: {\tt <http://www.emc.com/leadership/digital-universe/2014iview/executive-summary.htm>}.
\bibitem{teorieInformace} VAJDA, Igor. {\em Teorie informace.} Vyd. 1. Praha: Vydavatelství ČVUT, 2004. 109 s. ISBN 80-01-02986-7.
\bibitem{w3cxml} W3C. {\em Extensible Markup Language (XML) 1.0 (Fifth Edition).} 26. listopadu 2008, [online], [cit. 26. listopadu 2014]. Dostupné na: {\tt <http://www.w3.org/TR/2008/REC-xml-20081126/>}.
\end{thebibliography}
