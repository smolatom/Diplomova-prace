\chapter{Vlastní implementace vybraných kompresních algoritmů}

\section{Technická specifikace}
\subsection{Vývojové prostředí a programovací jazyk}
Na základě osobních preferencí jsem zvolil programovací jazyk C\# verze 5.0, který poskytuje sofistikované a otestované nástroje pro práci s XML i JSON, a vývojové prostředí Microsoft Visual Studio Ultimate 2013, které poskytuje plnou podporu pro práci se zvoleným jazykem.

Dalšími použitými nástroji jsou framework NUnit verze 2.6 pro účely testování, ReSharper 8.0.1 pro správu testů a refaktorování\footnote{Refaktorování je proces úprav kódu vedoucí ke zlepšení jeho struktury, ale nemající vliv na funkčnost.} a verzovací systém Git, jehož podpora je ve Visual Studiu přímo implementována.

\subsection{Implementace a testování}
Při implementaci jednotlivých algoritmů jsem postupoval metodou vývoje řízeného testy (aglicky Test Driven Development). Při té programátor nejprve napíše neprocházející test, který definuje výchozí a požadovaný stav, a až poté implementuje logiku tak, aby test prošel. Díky vhodně napsané sadě testů je možné velice rychle a snadno odhalit chybu při další implementaci nebo refaktorování.

\subsection{Projekt}
Výstupem mé práce je knihovna tří vybraných kompresních algoritmů, které komprimují jeden zadaný soubor. Struktura knihovny je zobrazena na obrázku \ref{strukturaKnihovny}. Každému algoritmu odpovídá jedna příznačně pojmenovaná třída, jejíž veřejné rozhraní poskytuje metody \texttt{Compress} a \texttt{Decompress}. Tyto metody na vstupu očekávají cestu k souboru, který má být zpracován, a po úspěšném zpracování všech dat na stejném umístění vytvoří soubor s~komprimovanými/dekomprimovanými daty.

\begin{figure}[!h]
\centering
\includegraphics[trim=70 875 415 55, clip, angle=0, width=150mm]{strukturaKnihovny}
\caption{Struktura knihovny kompresních algoritmů}
\label{strukturaKnihovny}
\end{figure}


\section{Vybrané algoritmy}
\subsection{Adaptivní Huffmanovo kódování}
Ze statistických kompresních metod jsem zvolil Vitterovu verzi adaptivního Huffmanova kódování, která je považována za složitější na implementaci, ale proti jiným verzím staví vyváženější stromy a tedy generuje kratší kódová slova. Uživatel má k dispozici třídu \texttt{AdaptiveHuffman} s veřejnými metodami \texttt{Compress} a \texttt{Decompress}. 

Jádrem algoritmu je ale třída \texttt{VitterTree}, která implementuje logiku chování stromu. Pro každý přečtený znak je při kompresi zavolána jeho metoda \texttt{AddChar} (viz algoritmus 1), která provede aktualizaci stromu dle algoritmu 2 a vrátí binární kód znaku ve stromu.

\begin{table}[!h]
\centering
\begin{tabular}{|l|}
\hline
\textbf{Algoritmus 1:} \texttt{AddChar}\\
\hline
\textbf{Vstupní parametry:} ASCII kód znaku \texttt{Z}\\
\textbf{Výstupní parametry:} binární kód vloženého znaku\\
\hline
\texttt{ 1} \texttt{if} strom obsahuje uzel se znakem \texttt{Z}\\
\texttt{ 2} \texttt{begin}\\
\texttt{ 3} \hspace*{5mm}vrať kód uzlu se znakem \texttt{Z}\\
\texttt{ 4} \hspace*{5mm}\texttt{UpdateTree(\texttt{Z})}\\
\texttt{ 5} \texttt{end}\\
\texttt{ 6} \texttt{else}\\
\texttt{ 7} \texttt{begin}\\
\texttt{ 8} \hspace*{5mm}vrať kód uzlu NYT\\
\texttt{ 9} \hspace*{5mm}vrať sedmibitový ASCII kód znaku \texttt{Z}\\
\texttt{10} \hspace*{5mm}\texttt{UpdateTree(\texttt{Z})}\\
\texttt{11} \texttt{end}\\
\hline
\end{tabular}
\end{table}

Pro účely dekomprese je strom implementován jako deterministický konečný automat. Obsahuje \texttt{ukazatel}, který udržuje referenci na jeden z uzlů (stav automatu). Na počátku \texttt{ukazatel} ukazuje na kořen a s každým přečteným bitem komprimované zprávy se přesune do následujícího uzlu dle přechodové funkce. Je-li přečten kód uzlu obsahujícího znak, je tento znak vrácen a strom aktualizován. Je-li přečten kód uzlu NYT, načte algoritmus dalších 7 bitů, které reprezentují ASCII kód nového znaku, ten je vložen do stromu a~vrácen. Zpracování jednoho bitu kódu provádí metoda \texttt{PushBit} (viz algoritmus 3).

\newpage
Při návrhu tohoto algoritmu jsem se snažil o dosažení co nejvyššího kompresního poměru, bohužel jsem tím, ale i díky nevědomosti, absolutně zanedbal správu paměti. Na tento nedostatek mne neupozornily ani jednotkové testy, které nejsou napsané pro řetězce dostatečně dlouhé na to, aby se únik paměti projevil. Algoritmus, tak jak je implementován, má problémy s pamětí i pro menší soubory a není ho tedy dost dobře možné použít při závěrečném porovnávání.

\begin{table}[!h]
\centering
\begin{tabular}{|l|}
\hline
\textbf{Algoritmus 2:} \texttt{UpdateTree}\\
\hline
\textbf{Vstupní parametry:} znak \texttt{Z}\\
\textbf{Výstupní parametry:} nemá\\
\hline
\texttt{ 1} uzel \texttt{U :=} uzel se znakem \texttt{Z}\\
\texttt{ 2} \texttt{while} uzel \texttt{U} není kořen\\
\texttt{ 3} \texttt{begin}\\
\texttt{ 4} \hspace*{5mm}\texttt{if} v bloku existuje uzel s vyšším indexem než má uzel \texttt{U}\\
\texttt{ 5} \hspace*{5mm}\texttt{begin}\\
\texttt{ 6} \hspace*{10mm}prohoď uzel \texttt{U} s uzlem s nejvyšším indexem v bloku\\
\texttt{ 7} \hspace*{5mm}\texttt{end}\\
\texttt{ 8} \hspace*{5mm}inkrementuj počet výskytů v uzlu \texttt{U}\\
\texttt{ 9} \hspace*{5mm}\texttt{U :=} předek uzlu \texttt{U}\\
\texttt{10} \texttt{end}\\
\hline
\end{tabular}
\end{table}


\begin{table}[!h]
\centering
\begin{tabular}{|l|}
\hline
\textbf{Algoritmus 3:} \texttt{PushBit}\\
\hline
\textbf{Vstupní parametry:} jeden bit kódu \texttt{B}\\
\textbf{Výstupní parametry:} přečtený znak, byl-li v tomto kroku přečten\\
\hline
\texttt{ 1} \texttt{if} čten znak nevyskytující se ve stromě\\
\texttt{ 2} \texttt{begin}\\
\texttt{ 3} \hspace*{5mm}kumuluj 7 bitů\\
\texttt{ 4} \hspace*{5mm}\texttt{if} nakumulováno 7 bitů\\
\texttt{ 5} \hspace*{5mm}\texttt{begin}\\
\texttt{ 6} \hspace*{10mm}vrať přečtený znak \texttt{Z}\\
\texttt{ 7} \hspace*{10mm}\texttt{AddChar(Z)}\\
\texttt{ 8} \hspace*{10mm}přesměruj \texttt{ukazatel} na kořen\\
\texttt{ 9} \hspace*{5mm}\texttt{end}\\
\texttt{10} \texttt{end}\\
\texttt{11} \texttt{else}\\
\texttt{12} \texttt{begin}\\
\texttt{13} \hspace*{5mm}dle \texttt{B} přesměruj \texttt{ukazatel} na jeho potomka\\
\texttt{14} \hspace*{5mm}\texttt{if} ukazatel je uzel obsahující znak\\
\texttt{15} \hspace*{5mm}\texttt{begin}\\
\texttt{16} \hspace*{10mm}vrať znak \texttt{Z} v uzlu\\
\texttt{17} \hspace*{10mm}\texttt{AddChar(Z)}\\
\texttt{18} \hspace*{10mm}přesměruj \texttt{ukazatel} na kořen\\
\texttt{19} \hspace*{5mm}\texttt{end}\\
\texttt{20} \hspace*{5mm}\texttt{else if} ukazatel je uzel NYT\\
\texttt{21} \hspace*{5mm}\texttt{begin}\\
\texttt{22} \hspace*{10mm}přesměruj \texttt{ukazatel} na kořen\\
\texttt{23} \hspace*{10mm}nastav stav stromu na čtení nového znaku\\
\texttt{24} \hspace*{5mm}\texttt{end}\\
\texttt{25} \texttt{end}\\
\hline
\end{tabular}
\end{table}

\subsection{JSONH}
Abych mohl ve staticky typovaném jazyce C\# zpracovat libovolná data ve formátu JSON, tedy pracovat s nimi jako s objekty různých typů, využívám typ \texttt{dynamic}. Ten umožňuje provést typovou kontrolu až v době běhu (run time), díky čemuž se s JSON v jazyce C\# pracuje částečně podobně jako v JavaScriptu. %Po zvládnutí transformace dat na objekty je implementace algoritmu jednoduchá.

Kompresní část algoritmu předpokládá, že všechny objekty v poli budou stejného typu. Komprese je závislá na struktuře objektů. Aby bylo dosaženo významného kompresního poměru, měly by objekty mít více atributů s jednoduchými hodnotami typu číslo, řetězec, \texttt{true}, \texttt{false} nebo \texttt{null} než méně atributů se složitými hodnotami typu pole nebo objekt. Metoda \texttt{Compress} třídy \texttt{JsonH} je popsána v algoritmu 4.

\begin{table}[!h]
\centering
\begin{tabular}{|l|}
\hline
\textbf{Algoritmus 4:} \texttt{Compress}\\
\hline
\textbf{Vstupní parametry:} cesta k souboru s JSON\\
\textbf{Výstupní parametry:} nemá\\
\hline
\texttt{ 1} načti JSON\\
\texttt{ 2} dekóduj JSON a vlož ho do dynamického pole \texttt{P}\\
\texttt{ 3} z prvního objektu v poli \texttt{P} načti klíče do kolekce \texttt{K}\\
\texttt{ 4} vytvoř prázdnou kolekci hodnot \texttt{H}\\
\texttt{ 5} \texttt{foreach} prvek \texttt{p} v poli \texttt{P}\\
\texttt{ 6} \texttt{begin}\\
\texttt{ 7} \hspace*{5mm}\texttt{foreach} klíč \texttt{k} v kolekci \texttt{K}\\
\texttt{ 8} \hspace*{5mm}\texttt{begin}\\
\texttt{ 9} \hspace*{10mm}serializuj hodnotu pro klíč \texttt{k} v prvku \texttt{p} do kolekce \texttt{H}\\
\texttt{10} \hspace*{5mm}\texttt{end}\\
\texttt{11} \texttt{end}\\
\texttt{12} vytvoř dynamické pole \texttt{D} obsahující počet prvků v kolekci \texttt{K},\\
\verb+  + prvky kolekce \texttt{K} a prvky kolekce \texttt{H}\\
\texttt{13} serializuj pole \texttt{D} do souboru\\
\hline
\end{tabular}
\end{table}

Dekompresní algoritmus je implementován v metodě \texttt{Decompress} třídy \texttt{JsonH}. Jeho implementace je velmi jednoduchá, což dosvědčuje i algoritmus 5.

\begin{table}[!h]
\centering
\begin{tabular}{|l|}
\hline
\textbf{Algoritmus 5:} \texttt{Decompress}\\
\hline
\textbf{Vstupní parametry:} cesta k souboru s JSON\\
\textbf{Výstupní parametry:} nemá\\
\hline
\texttt{ 1} načti JSON\\
\texttt{ 2} dekóduj JSON a vlož ho do dynamického pole \texttt{P}\\
\texttt{ 3} z prvního prvku v poli \texttt{P} načti počet klíčů do proměnné \texttt{n}\\
\texttt{ 4} z následujících \texttt{n} prvků vytvoř šablonu objektu \texttt{S}\\
\texttt{ 5} ze zbývajících prvků vytvoř pole objektů \texttt{D} dle šablony \texttt{S}\\
\texttt{ 6} serializuj pole \texttt{D} do souboru\\
\hline
\end{tabular}
\end{table}

Tento algoritmus je implementován naprosto jinak než \texttt{AdaptiveHuffman} a problémy s~pamětí netrpí. Mohl jsem ho tedy pro testovací soubory obsahující homogenní kolekce použít při závěrečném porovnávání.

\subsection{LZ77}
Poučen z chyb udělaných při návrhu a vývoji adaptivního Huffmanova kódování jsem upustil od snahy dosáhnout co nejvyššího kompresního poměru a zaměřil se na správu paměti. Zakódovaná i dekódovaná zpráva jsou do soubory ukládány po menších blocích, které mají určenou maximální délku. Data zakódované zprávy jsou z důvodů dekomprese v bloku uvozena počtem použitých bitů.

Třída \texttt{LZ77}, která se stará o kompresi i dekompresi, zpracovává postupně jednotlivé bloky. Komprese jednoho bloku je naznačena v algoritmu 6. V případě, že je zpracován celý blok, dojde k načtení následujícího a proces komprese se opakuje. Implementační detaily zachování konzistence mezi načítáním bloků zde neuvádím.

\begin{table}[!h]
\centering
\begin{tabular}{|l|}
\hline
\textbf{Algoritmus 6:} \texttt{compressBlock}\\
\hline
\textbf{Vstupní parametry:} blok dat \texttt{B} určený ke kompresi\\
\textbf{Výstupní parametry:} nemá\\
\hline
\texttt{ 1} \texttt{while} není dosaženo konce bloku \texttt{B}\\
\texttt{ 2} \texttt{begin}\\
\texttt{ 3} \hspace*{5mm}najdi nejdelší frázi ve slovníku\\
\texttt{ 4} \hspace*{5mm}zakóduj trojici offset a délka nalezené fráze a další znak z bloku B\\
\texttt{ 5} \hspace*{5mm}posuň okno o počet zakódovaných znaků vpravo\\
\texttt{ 6} \texttt{end}\\
\hline
\end{tabular}
\end{table}

Dekompresní část je v podstatě inverzí ke kompresi. Tak, jak byly při kompresi vytvářeny kódovací trojice, jsou nyní čteny a dekódovány. Rozdílem proti kompresi je, že nové fráze vkládané na konec slovníku je nutné nejprve dekódovat. Stav slovníku je ale v každém kroku shodný.