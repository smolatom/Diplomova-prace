\documentclass[a4paper,11pt]{report}
\usepackage[czech]{babel}	%česky psaná práce
\usepackage[utf8]{inputenc}	%znaková sada utf8

\usepackage{amsmath} % baíček pro pokročilou matem. sazbu
\usepackage{amsthm}
\usepackage{epsfig} % balíčky pro vkládání grafických souborů typu EPS
\usepackage{graphicx} % balíček pro vložení grafických souborů
\usepackage{amsfonts}
\usepackage{amssymb}

%\usepackage{setspace} 

\usepackage{latexsym}
\usepackage[usenames,dvipsnames]{color}
\usepackage{multirow}

%\usepackage{regexpatch}
%\makeatletter
%\xpatchparametertext\@cline{-}{\cA-}{}{}
%\makeatothe

\graphicspath{{pics/}}

\oddsidemargin=10mm   % levý okraj větší (kvůli vazbě)
\topmargin=-15mm      % horní okraj trochu menší
\textwidth=150mm      % výška textu na stránce
\textheight=240mm     % "výška" textu na stránce

\pagenumbering{arabic} % číslování stránek arabskými číslicemi
\pagestyle{plain}      % stránky číslované dole uprostřed

\parindent=0pt % odsazení 1. řádku odstavce
\parskip=7pt   % mezera mezi odstavci
\frenchspacing % aktivuje použití některých českých typografických pravidel


\newcommand{\xml}{XML}
\newcommand{\json}{JSON}


% definice makra pro české uvozovky:
\def\bq{\mbox{\kern.1ex\protect\raisebox{-1.3ex}[0pt][0pt]{''}\kern-.1ex}}
\def\eq{\mbox{\kern-.1ex``\kern.1ex}}
\def\ifundefined#1{\expandafter\ifx\csname#1\endcsname\relax }%
\ifundefined{uv}%
        \gdef\uv#1{\bq #1\eq}
\fi
% konec .... použití makra pro psaní českých uvozovek: \uv{text uvnitř uvozovek}

% zde jsou zavedeny některé "konstanty" - můžete, resp. musíte je ZMĚNIT 
\newcommand{\cvut}{České vysoké učení technické v~Praze}
\newcommand{\fjfi}{Fakulta jaderná a fyzikálně inženýrská}
\newcommand{\kse}{Katedra softwarového inženýrství}
\newcommand{\obor}{Aplikace softwarového inženýrství}

\newcommand{\nazevcz}{Porovnání účinnosti komprese dat ve formátech XML a JSON}        % zde VYPLŇTE český název práce (přesně podle zadání!)

%The proposal of a service programme for smart consumption meters
\newcommand{\nazeven}{Comparison of the effectiveness of data compression in XML and JSON format}     % zde VYPLŇTE anglický název práce (přesně podle zadání!)
\newcommand{\autor}{Bc. Tomáš~Smola}           % zde VYPLŇTE své jméno a příjmení
\newcommand{\rok}{2015}                % zde VYPLŇTE rok odevzdání, např. 2006
\newcommand{\vedouci}{Ing.~Tomáš~Liška,~Ph.D.}         % zde VYPLŇTE jméno a příjmení vedoucího práce, včetně titulů
                                                               % např. Doc. Ing. Ivo Malý, Ph.D.
\newcommand{\pracovisteVed}{\kse, \fjfi, \cvut} % zde VYPLŇTE pracoviště vedoucího práce, je-li jiné než KSE FJFI ČVUT

\newcommand{\konzultant}{---} % POKUD MÁTE určeného konzultanta, NAPIŠTE jeho jméno a příjmení
\newcommand{\pracovisteKonz}{} % POKUD MÁTE konzultanta, NAPIŠTE jeho pracoviště

\newcommand{\klicova}{Klíčová slova}   % zde NAPIŠTE česky max. 5 klíčových slov
\newcommand{\keyword}{Key words}       % zde NAPIŠTE anglicky max. 5 klíčových slov (přeložte z češtiny)
\newcommand{\abstrCZ}{Abstrakt} % zde NAPIŠTE abstrakt včeštině
\newcommand{\abstrEN}{Abstract}                  % zde NAPIŠTE abstrakt v angličtině
\begin{document}

% 1. strana -- na následujících 30 řádků NESAHEJTE!!!  Generují se AUTOMATICKY 
\thispagestyle{empty}

\begin{center}
    {\Large \bf \cvut\\[2mm] \fjfi }
    \vspace{10mm}

    \begin{tabular}{c}
    {\bf \kse}\\
    {\bf Obor: \obor}\\
    \end{tabular}

   % logo CVUT
   %\vspace{10mm} \epsfysize=25mm \epsffile{cvut-logo-bw} \vspace{15mm}

   \vspace{10mm} \includegraphics[angle=0, width=25mm]{cvut-logo-bw} \vspace{15mm}

   {\Huge \bf \nazevcz\par \vspace{5mm} \nazeven }

   \vspace{15mm}
   {\Large DIPLOMOVÁ PRÁCE}

   \vfill
   {\large
    \begin{tabular}{rl}
    Vypracoval: & \autor\\
    Vedoucí práce: & \vedouci\\
    Rok: & \rok
    \end{tabular}
   }
\end{center}

% 2. strana: zadání práce  
%   Před svázáním místo téhle stránky vložte zadání práce s podpisem děkana
\newpage  % SEM NESAHEJTE!
\thispagestyle{empty} Před svázáním místo téhle stránky \fbox{vložte zadání práce} s podpisem
děkana (bude to jediný oboustranný list ve Vaší práci) !!!!

% 3. strana: prohlášení
\newpage % SEM NESAHEJTE!
\thispagestyle{empty}  % SEM NESAHEJTE!

~ % SEM NESAHEJTE!
\vfill % prázdné místo. SEM NESAHEJTE!

{\bf Prohlášení} % SEM NESAHEJTE!

\vspace{0.5cm} % vertikální mezera. SEM NESAHEJTE!
Prohlašuji, že jsem svou diplomovou práci vypracoval samostatně a použil jsem pouze podklady
(literaturu, projekty, SW atd.) uvedené v přiloženém seznamu.

\vspace{5mm}V Praze dne ....................\hfill  % SEM NESAHEJTE!
    \begin{tabular}{c}                               % SEM NESAHEJTE!
    ........................................\\       % SEM NESAHEJTE!
    \autor                                           % SEM NESAHEJTE!
    \end{tabular}                                    % SEM NESAHEJTE!

% 4. strana: poděkování -- UPRAVTE JMÉNO
\newpage
\thispagestyle{empty}

~
\vfill % prázdné místo

{\bf Poděkování}

\vspace{5mm} % vertikální mezera
Děkuji Ing. Tomáši Liškovi, Ph.D. za vedení mé diplomové práce a za podnětné návrhy, které ho obohatily.

\begin{flushright}
\autor
\end{flushright}  % <------- tady končí stránka s poděkováním

% 5. strana: abstrakt atp. Je generován AUTOMATICKY podle údajů na začátku souboru)
\newpage   % SEM NESAHEJTE!
\thispagestyle{empty}   % SEM NESAHEJTE!

% příprava:    (na následujících 8 řádků NESAHEJTE!)
\newbox\odstavecbox
\newlength\vyskaodstavce
\newcommand\odstavec[2]{%
    \setbox\odstavecbox=\hbox{%
         \parbox[t]{#1}{#2\vrule width 0pt depth 4pt}}%
    \global\vyskaodstavce=\dp\odstavecbox
    \box\odstavecbox}
\newcommand{\delka}{120mm} % šířka textu ve 2. sloupci tabulky

% použití přípravy:    % dovnitř "tabular" vůbec NESAHEJTE!
\begin{tabular}{ll}
  {\em Název práce:} & ~ \\
  \multicolumn{2}{l}{\odstavec{\textwidth}{\bf \nazevcz}} \\[5mm]
  {\em Autor:} & \autor \\[5mm]
  {\em Obor:} & \obor \\
  {\em Druh práce:} & Diplomová práce \\[5mm]
  {\em Vedoucí práce:} & \odstavec{\delka}{\vedouci \\ \pracovisteVed} \\[5mm]
  {\em Konzultant:} & \odstavec{\delka}{\konzultant \\ \pracovisteKonz} \\[5mm]
  \multicolumn{2}{l}{\odstavec{\textwidth}{{\em Abstrakt:} ~ \abstrCZ \\ }} \\[5mm]
  {\em Klíčová slova:} & \odstavec{\delka}{\klicova} \\[20mm]

  {\em Title:} & ~\\
  \multicolumn{2}{l}{\odstavec{\textwidth}{\bf \nazeven}}\\[5mm]
  {\em Author:} & \autor \\[5mm]
  \multicolumn{2}{l}{\odstavec{\textwidth}{{\em Abstract:} ~ \abstrEN \\ }} \\[5mm]
  {\em Key words:} & \odstavec{\delka}{\keyword}
\end{tabular}

% 6. strana: obsah práce je generován AUTOMATICKY
\newpage  % SEM NESAHEJTE!
\setcounter{page}{1}
\pagenumbering{roman}
\tableofcontents % SEM NESAHEJTE!

%  7.strana: zde zadání SAMOTNÉ PRÁCE  
%                                 - text se vkládá Z EXTERNÍCH SOUBORŮ
%                                   (můžete ho také napsat přímo sem => smažte každý \input{...})
\newpage % SEM NESAHEJTE!

\chapter*{Úvod} \addcontentsline{toc}{chapter}{Úvod} % SEM NESAHEJTE!
%Závěrečnou diplomovou práci ke studijnímu oboru Aplikace softwarového inženýrství s~ná\-zvem Porovnání účinnosti komprese dat ve formátech XML a JSON jsem si vybral z~důvodu aktuálnosti, XML a JSON jsou v současnosti jedny z nejpoužívanějších textových datových formátů, a také proto, že toto téma velmi dobře propojuje teoretické znalosti získané při studiu s praktickými zkušenostmi v oboru softwarového inženýrství.

Dle výzkumu International Data Corporation (IDC) The Digital Universe of Opportunities: Rich Data and the Increasing Value of the Internet of Things \cite{idc} bylo pouze v roce 2014 vytvořeno a zkonzumovýno 2837~EB (exabytů) dat. Ze závěrů vyplývá, že se toto číslo každé dva roky zdvojnásobí, tedy v roce 2020 to bude již přibližně 40000 EB. Takové množství dat klade enormní požadavky na přenosové kanály a datová úložiště. Jako příklad mohou sloužit miliony shlédnutí oblíbených videí na serveru youtube.com. Pouze jedna sekunda videa v nekomprimovaném formátu CCIR 601 zabere více než 20 MB, tímto způsobem by zmiňovaná služba nemohla fungovat.

Vzhledem k tomu, že jsou technologie omezeny současnými možnostmi, znalostmi a také fyzikálními limity, je nutné hledat řešení jinde než v jejich zlepšování. Zde přichází na řadu komprese dat jako účinná metoda snížení velikosti objemu přenášených a ukládaných dat. Ve své práci bych čtenáře rád seznámil se základními principy komprese a vybranými kompresními algoritmy. Hlavním cílem je ale zodpovědět otázku, zda je možné dosáhnout dalších úspor volbou XML nebo JSON formátu a vhodného algoritmu, který využije znalosti struktury datového formátu.

Tato práce soustřeďující se na využití znalosti struktury dat při kompresi se skládá celkem z 8 kapitol. Postupně jsou čtenáři představeny porovnávané datové formáty XML a~JSON, základní techniky komprese, existující kompresní algoritmy, mnou implementované algoritmy a na závěr porovnání vybraných algoritmů na testovacích datech.

První kapitola otevírá teoretickou část práce a popisuje porovnávané formáty a to jak z~pohledu důvodů vzniku a použitelnosti, tak i z pohledu syntaktické analýzy. Zápis dat pomocí obou formátů je znázorněn na krátkých příkladech. Tyto znalosti jsou stěžejní pro praktickou část práce.

Druhá, třetí a čtvrtá kapitola jsou věnovány základním pojmům kompresního procesu a~vybraným technikám bezeztrátové komprese. Pomocí příkladů jsou zde čtenáři názorně vysvětleny na příkladech způsoby přístupu ke kompresi známých algoritmů.

V kapitolách 5 a 6 jsou představeny algoritmy využívající znalosti struktury dat v XML a JSON. Důraz je kladen především na popis využití této znalosti a způsobu zpracování dat. Tímto je uzavřena teoretická část práce a získané poznatky jsou aplikovány v části praktické.

Praktická část se skládá ze dvou kapitol. V první popisuji způsob implementace vybraných kompresních algoritmů, který by měl vnést hlubší pohled do problematiky a odkrýt potenciální slabá místa. Poslední kapitola je věnována vzájemnému porovnání účinnosti komprese na data ve formátech XML a JSON. Hodnoty naměřené při testování jsou zobrazeny v přehledných grafech s vysvětlením příčin, které k tomuto výsledku mohly vést.

V závěru práce jsou na základě naměřených hodnot shrnuty a vyhodnoceny poznatky o~účinnosti komprese využívající znalosti struktury datového formátu.  % text úvodu (bez nadpisu) je vkládán z jiného souboru; lze upravit jméno souboru, resp. smazat tento řádek a text úvodu napsat přímo sem (ale znepřehlední se to)

\setcounter{page}{1}
\pagenumbering{arabic}

Závěrečnou diplomovou práci ke studijnímu oboru Aplikace softwarového inženýrství s~ná\-zvem Porovnání účinnosti komprese dat ve formátech XML a JSON jsem si vybral z~důvodu aktuálnosti, XML a JSON jsou v současnosti jedny z nejpoužívanějších textových datových formátů, a také proto, že toto téma velmi dobře propojuje teoretické znalosti získané při studiu s praktickými zkušenostmi v oboru softwarového inženýrství.

Dle výzkumu International Data Corporation (IDC) The Digital Universe of Opportunities: Rich Data and the Increasing Value of the Internet of Things \cite{idc} bylo pouze v roce 2014 vytvořeno a zkonzumovýno 2837~EB (exabytů) dat. Ze závěrů vyplývá, že se toto číslo každé dva roky zdvojnásobí, tedy v roce 2020 to bude již přibližně 40000 EB. Takové množství dat klade enormní požadavky na přenosové kanály a datová úložiště. Jako příklad mohou sloužit miliony shlédnutí oblíbených videí na serveru youtube.com. Pouze jedna sekunda videa v nekomprimovaném formátu CCIR 601 zabere více než 20 MB, tímto způsobem by zmiňovaná služba nemohla fungovat.

Vzhledem k tomu, že jsou technologie omezeny současnými možnostmi, znalostmi a také fyzikálními limity, je nutné hledat řešení jinde než v jejich zlepšování. Zde přichází na řadu komprese dat jako účinná metoda snížení velikosti objemu přenášených a ukládaných dat. Ve své práci bych čtenáře rád seznámil se základními principy komprese a vybranými kompresními algoritmy. Hlavním cílem je ale zodpovědět otázku, zda je možné dosáhnout dalších úspor volbou XML nebo JSON formátu a vhodného algoritmu, který využije znalosti struktury datového formátu.

Tato práce soustřeďující se na využití znalosti struktury dat při kompresi se skládá celkem z 8 kapitol. Postupně jsou čtenáři představeny porovnávané datové formáty XML a~JSON, základní techniky komprese, existující kompresní algoritmy, mnou implementované algoritmy a na závěr porovnání vybraných algoritmů na testovacích datech.

První kapitola otevírá teoretickou část práce a popisuje porovnávané formáty a to jak z~pohledu důvodů vzniku a použitelnosti, tak i z pohledu syntaktické analýzy. Zápis dat pomocí obou formátů je znázorněn na krátkých příkladech. Tyto znalosti jsou stěžejní pro praktickou část práce.

Druhá, třetí a čtvrtá kapitola jsou věnovány základním pojmům kompresního procesu a~vybraným technikám bezeztrátové komprese. Pomocí příkladů jsou zde čtenáři názorně vysvětleny na příkladech způsoby přístupu ke kompresi známých algoritmů.

V kapitolách 5 a 6 jsou představeny algoritmy využívající znalosti struktury dat v XML a JSON. Důraz je kladen především na popis využití této znalosti a způsobu zpracování dat. Tímto je uzavřena teoretická část práce a získané poznatky jsou aplikovány v části praktické.

Praktická část se skládá ze dvou kapitol. V první popisuji způsob implementace vybraných kompresních algoritmů, který by měl vnést hlubší pohled do problematiky a odkrýt potenciální slabá místa. Poslední kapitola je věnována vzájemnému porovnání účinnosti komprese na data ve formátech XML a JSON. Hodnoty naměřené při testování jsou zobrazeny v přehledných grafech s vysvětlením příčin, které k tomuto výsledku mohly vést.

V závěru práce jsou na základě naměřených hodnot shrnuty a vyhodnoceny poznatky o~účinnosti komprese využívající znalosti struktury datového formátu.

% Následují texty kapitol. Smažete-li následující řádky (text chcete psát přímo sem), nezapomeňte na CHAPTER!!!
\chapter{Obecné seznámení s formáty XML a JSON}
V této kapitole seznámím čtenáře se značkovacím jazykem XML a následně s JSONem, formátem pro výměnu dat. Mým cílem je popsat základní charakteristiky a syntaxi obou formátů tak, abych byl já, a následně i čtenář, schopen pochopit v kapitole \ref{kapitolaSpecifickeAlgoritmy} principy a výhody vybraných algoritmů využívajících znalosti struktury datových souborů.

\section{XML}
Na základě značkovacího jazyka SGML (Standard Generalized Markup Language), jehož obecnost činí úplnou implementaci velmi náročnou, vznikl vybráním nejpoužívanějších možností nový značkovací jazyk XML (eXtensible Markup Language), je tedy podmnožinou jazyka SGML. XML je obecný a otevřený, jeho vývoj a standardizaci realizovalo konsorcium W3C (World Wide Web Consorcium) \cite{w3cxml}. XML umožňuje snadné vytváření konkrétních značkovacích jazyků pro popis dokumentů a dat ve standardizované, textově orientované podobě. 

\subsection{Charakteristika}


\subsection{Syntaktická analýza}
V podstatě jde o textový dokument, je tvořen posloupností Unicode\footnote{Unicode je standard pro konzistentní kódování, reprezentaci a manipulaci znaků většiny světových abeced.} znaků, ve kterém se rozlišují dva základní prvky: elementy neboli značky a obsah. Při práci s XML je nutné mít na paměti, že je na dodržení syntaxe kladen velmi velký důraz. Při dodržení správného způsobu zápisu a pravidel, která budou popsána níže, lze dokument považovat za tzv. \textit{well-formed XML} \cite{w3cxml}.

\subsubsection{Element}
Základním prvkem každého XML dokumentu je element, který je vyznačen pomocí takzvaných tagů\footnote{Tag definuje formu části textu.}, mezi které může být vložen obsah. Počáteční i ukončující tag je dle definice \cite{w3cxml} složen z dvojice znamének \texttt{<} (menší než) a \texttt{>} (větší než), mezi kterými je zapsán název tagu a volitelně i atributy. Ukončovací tag má navíc před svým názvem znak \texttt{/} (lomeno). Při správné aplikaci pravidel může vypadat element například následujícím způsobem:
$$\texttt{<název\_elementu název\_atributu=\textquotedblright hodnota atributu\textquotedblright></název\_elementu>}.$$
V případě, že element neobsahuje žádný obsah, lze ho zkráceně zapsat jako tzv. prázdný element:
$$\texttt{<název\_prázdného\_elementu />}.$$
V případě nedodržení správné syntaxe může nastat problém při rozpoznávání zapsaných dat, což může mít za následek nekompatibilitu mezi různými systémy při výměně dat.

\subsubsection{Atribut}
Počáteční tag elementu může obsahovat atributy upřesňující jeho význam. Atribut je vždy složen ze svého názvu a hodnoty, které jsou odděleny znakem \texttt{=} (rovná se). Hodnota je navíc zapsána mezi dvojici znaků \texttt{\textquotedblright} (uvozovky) nebo \texttt{\textquoteright} (apostrof), přičemž hodnota může obsahovat jeden z těchto znaků tak, že se syntakticky nekříží. Následuje příklad atributu, jehož hodnota obsahuje znak \texttt{\textquoteright}:
$$\texttt{název\_atributu=\textquotedblright hodnota atributu osahující znak \textquoteright\ (apostrof)\textquotedblright}.$$

\subsubsection{Obsah}
Vše, co není tagem, je v dokumentu považováno za obsah. Kromě obyčejného textu mohou být obsahem další vnořené elementy, komentáře, instrukce pro zpracování, reference a další. Vzhledem k tomu, že určité znaky mají v syntaxi XML speciální význam (např. \texttt{<}, \texttt{>}), využívají se pro jejich zápis znakové entity\footnote{Pomocí znakových entit (sekvence znaků) lze zapsat znaky, které neobsahuje zvolená znaková sada, nebo mají v použitém kontextu speciální význam.}. Úplný výčet toho, co může XML dokument obsahovat, je včetně pravidel definován v \cite{w3cxml}.

\subsection{Parsování}
\subsection{Výhody a nevýhody}
\section{JSON}
JSON neboli JavaScript Object Notation je odlehčený způsob zápisu (formátování) dat. Tento textový formát je nezávislý na počítačové platformě a je čitelný pro člověka. JSON je založen na dvou univerzálních datových strukturách: kolekce dvojic klíč/hodnota a seřazený seznam hodnot, které podporují v nějaké formě asi všechny známé moderní programovací jazyky. Díky těmto vlastnostem se JSON stal velmi oblíbeným formátem pro vzájemnou výměnu dat.

\subsection{Charakteristika}

\subsection{Syntaktická analýza}
Jak již bylo zmíněno, je JSON textový formát a je tedy posloupností tokenů tvořených z Unicode znaků. Sada tokenů obsahuje šest strukturálních tokenů: \texttt{[} (levá hranatá závorka), \texttt{\{} (levá složená závorka), \texttt{]} (pravá hranatá závorka), \texttt{\}} (pravá složená závorka), \texttt{:} (dvojtečka) a \texttt{,} (čárka); dále obsahuje znakové řetězce, čísla a tři doslovné tokeny: \texttt{true}, \texttt{false} a \texttt{null}.

\subsubsection{Hodnoty}
Za hodnotu je v JSONu považován objekt, pole, číslo, řetězec, \texttt{true}, \texttt{false}, nebo \texttt{null}.

\begin{figure}[!htb]
\centering
\includegraphics[trim=0 260 90 30, clip, angle=0, width=150mm]{hodnota}
\caption{Struktura hodnoty}
\label{hodnota}
\end{figure}

\subsubsection{Objekty}
Objekt je reprezentován dvojicí složených závorek, uvnitř kterých je žádná nebo více dvojic klíč/hodnota, přičemž klíč je řetězec. Klíč a hodnota jsou odděleny dvojtečkou a jednotlivé dvojice odděluje čárka.

\begin{figure}[!htb]
\centering
\includegraphics[trim=70 430 80 70, clip, angle=0, width=150mm]{objekt}
\caption{Struktura objektu}
\label{objekt}
\end{figure}

\subsubsection{Pole}
Pole je složeno z dvojice hranatých závorek, mezi kterými může být nula nebo více seřazených hodnot, které jsou odděleny čárkou.

\begin{figure}[!htb]
\centering
\includegraphics[trim=60 445 90 55, clip, angle=0, width=150mm]{pole}
\caption{Struktura pole}
\label{pole}
\end{figure}

\subsubsection{Čísla}
Čísla jsou v desítkové soustavě (tedy číslice $0 - 9$) záporná čísla jsou uvozena znaménkem \texttt{-} (mínus), desetinná část je oddělena znaménkem \texttt{.} (tečka). Je možný i takzvaný vědecký zápis čísel s použitím symbolů \texttt{e} (malé e) nebo \texttt{E} (velké e) a volitelně lze použít u exponentu znaménka \texttt{+} (plus) nebo \texttt{-} (mínus).

\begin{figure}[!htb]
\centering
\includegraphics[trim=80 550 240 55, clip, angle=0, width=150mm]{cislo}
\caption{Struktura čísla}
\label{cislo}
\end{figure}

\subsubsection{Řetězce}
Řetězec je posloupnost Unicode znaků uvozená a zakončená znakem \texttt{\textquotedblright} (uvozovky). Mezi uvozovky mohou být zapsány všechny znaky kromě speciálních znaků, které jsou uvozeny tzv. únikovým znakem \texttt{\textbackslash} (zpětné lomítko), těmito znaky jsou například \texttt{\textquotedblright}, \texttt{\textbackslash}, \texttt{t} (znak tabulátoru), \texttt{n} (znak posunu kurzoru na nový řádek) a \texttt{r} (znak posunu kurzoru na začátek řádku, známý jako návrat vozíku) a další. Kompletní výčet je možné vidět na obrázku \ref{retezec} nebo v \cite{json}. Navíc je možné každý znak zapsat pomocí kombinace \texttt{\textbackslash u} a čtyřmístného hexadecimálního čísla odpovídajícího Unicode kódu požadovaného znaku. Řetězec obsahující pouze zpětné lomítko můžeme tedy zapsat následujícími způsoby: \texttt{\textquotedblright\textbackslash\textbackslash\textquotedblright}, \texttt{\textquotedblright\textbackslash u005c\textquotedblright} nebo \texttt{\textquotedblright\textbackslash u005C\textquotedblright} (u hexadecimálních čísel \texttt{A-F} nezáleží na velikosti).

\begin{figure}[!htb]
\centering
\includegraphics[trim=80 440 360 55, clip, angle=0, width=150mm]{retezec}
\caption{Struktura řetězce}
\label{retezec}
\end{figure}

\subsection{Parsování}
\subsection{Výhody a nevýhody}  % Obecné seznámení s formáty XML a JSON
\chapter{Komprese dat}
Komprese nebo také komprimace dat je taková transformace dat, která má za cíl úsporu zdrojů při ukládání nebo archivaci a nebo snížení datového toku při přenosu, to vše při současném zachování informace obsažené v datech. Jinými slovy jde o redukci velikosti datových souborů, jehož následkem je úspora paměťových či přenosových kapacit. Postup, při kterém z komprimovaných dat rekonstruujeme data originální, se nazývá dekomprese.

\section{Princip komprese dat}
\label{sekcePrincipKompreseDat}
Data velmi často obsahují tzv. redundantní\footnote{Redundance znamená informační nadbytek, např. vícenásobný výskyt slov v textu.} informaci, toho právě využívá komprese -- data jsou zpracována tak, aby byla redundance minimalizována. Jak lze vidět na obrázku \ref{kompreseDekomprese}, je na vstupní data použita operace komprese. Operací dekomprese dostaneme poté data rekonstruovaná -- v závislosti na použité kompresní metodě, respektive na požadavcích získáme buď data přesně odpovídající původním a nebo pouze částečná. Z tohoto hlediska rozlišujeme dva typy kompresních metod: ztrátové a bezeztrátové.

\begin{figure}[!htb]
\centering
\includegraphics[trim=0 610 60 55, clip, angle=0, width=150mm]{kompreseDekomprese}
\caption{Princip komprese}
\label{kompreseDekomprese}
\end{figure}


\section{Typy kompresních metod}
Jak název napovídá, při ztrátové kompresi ztratíme část informace obsaženou v původních datech, respektive jsou původní data pouze aproximována.  Toto nám nemusí vadit například u obrázků, zvuku a videí, kde je využito nedokonalosti lidských smyslů. Lidské ucho nedokáže například slyšet velmi vysoké frekvence. Má smysl v datech určených k poslechu zachovávat informaci, kterou nemůže člověk slyšet? Častá odpověď je \uv{ne}. Tohoto principu využívá mnoho kompresních metod, například známý zvukový formát MP3. Odstraněním \uv{nepotřebné} informace z dat je dosaženo ještě větší redukce objemu. 

Naopak v případě bezeztrátových metod je při kompresi zachována veškerá informace a při dekompresi
jsou rekonstruována původní data. Těchto metod se využívá převážně tam, kde není možné původní data jakkoliv pozměnit. Například data ve formátech XML a JSON, kterým se věnuji v této práci, si nemůžeme dovolit pozměnit (přestanou mít původní význam), nebo dokonce ztratit.

\section{Charakteristika komprese}
Kompresní algoritmy lze hodnotit z mnoha různých úhlů pohledu. Můžeme měřit složitost algoritmu, rychlost, jakou data komprimována a dekomprimována (to může být ovlivněno výkonem stroje, na kterém algoritmus běží), jak moc odpovídají rekonstruovaná data původním atd.
Jednou z nejčastějších charakteristik je, logicky ze smyslu komprese vy\-plý\-va\-jí\-cí, tzv. kompresní poměr, který vyjadřuje velikost komprimovaných dat vůči původním, lze ho zapsat následujícím  vztahem:
\begin{equation}
\texttt{kompresní poměr} = \frac{\texttt{délka původních dat}}{\texttt{délka komprimovaných dat}}.
\end{equation}

Další sledovanou charakteristikou je tzv. úspora místa, která je vyjádřena jako:
\begin{equation}
\texttt{úspora místa} = \texttt{1 - kompresní poměr$^{-1}$}.
\end{equation}

Mějme například 2D obrázek o velikosti $256\times256$ pixelů, který zabírá 65536 bytů. Obrázek zkomprimujeme a zabírá-li komprimovaná verze 16384 bytů, můžeme říct, že kompresní poměr je $4:1$ a úspora místa 75 \%. 

\section{Míra informace v datech}
  % Komprese dat
\chapter{Popis existujících kompresních algoritmů}  % Statistické techniky komprese
\chapter{Přehled existujících implementací kompresních algoritmů pro efektivní uchovávání dat ve formátu XML a JSON}  % Slovníkové techniky komprese
\chapter{Existující algoritmy pro kompresi XML}  % Existujících algoritmy pro kompresi XML
\chapter{Existující algoritmy pro kompresi JSON}
\label{kapitolaJsonAlgoritmy}

JSON byl navržen jako odlehčená varianta způsobu formátování dat proti XML, čímž byla odstraněna i redundance při použití počátečního a ukončovacího tagu. Po seznámení s jeho definicí a syntaxí je zřejmé, že zde již nezbývá moc prostoru k dalšímu odlehčení. Podíváme-li se ale na data zapsaná v tomto formátu, která mohou obsahovat například výstup SQL příkazu SELECT nad databází, můžeme vidět, že se zde některé prvky přece jen opakují. Jsou to klíče, včetně uvozovek, v jednotlivých objektech, které stačí zapsat pouze jednou. Tohoto poznatku využívají i dva vybrané algoritmy JSONH a CJSON, které jsou specifické tím, že výstupem z nich je validní JSON. Rád bych čtenáře upozornil, že algoritmů pro kompresi JSON neexistuje mnoho, resp. jsou si velice podobné myšlenkou, provedením i názvy.

S daty v tomto formátu nepracujeme jako s obyčejným textem, ale převedeme do paměti jako hodnoty (pole, objekty atd.), což odpovídá zpracování pomocí JavaScriptu. Tomu odpovídá i terminologie použitá v této kapitole, která byla popsána v části \ref{syntaxeJson}.

\section{JSONH}
\label{jsonh}
Tento algoritmus dovoluje komprimovat pouze homogenní kolekce dat, v terminologii JSON jde o pole objektů, které mají stejný počet klíčů se stejnými názvy. Homogenitu dat musí zaručit uživatel, algoritmus samotný toto nikterak neošetřuje. Autor projektu JSONH \cite{jsonh}, kde je algoritmus implementován, na serveru github.com uvádí, že data mohou být v některých případech zmenšena až na 30 \%.

\subsection{Vzorová data}
Mějme homogenní kolekci zaměstnanců firmy, ve které máme uložené databázové id, příjmení a pozici, na které zaměstnanec pracuje. Část této kolekce vypadá následujícím způsobem:

\begin{verbatim}
[ { "id" : 1, "name" : "Sánchez", "position" : "Manager" },
    { "id" : 2, "name" : "Duffy", "position" : "Programmer" },
    { "id" : 3, "name" : "Tamburello", "position" : "Programmer" } ]
\end{verbatim}

\subsection{Postup komprese}
\begin{enumerate}
\item Textová data jsou převedena na pole objektů.
\item \label{jsonhItem0} Z prvního objektu v poli jsou určeny klíče a jejich počet.
\item \label{jsonhItem1} Ze všech objektů v poli jsou postupně vyzvednuty hodnoty pro příslušné klíče, přičemž je zachováno pořadí klíčů, jak jsme jej získali v kroku \ref{jsonhItem0}.
\item Je vytvořeno pole, které obsahuje prvky: počet klíčů, seznam klíčů a hodnoty vyzvednuté v kroku \ref{jsonhItem1}.
\item Vytvořené pole serializujeme jako JSON do řetězce nebo souboru.
\end{enumerate}

Data po kompresi mají následující podobu:
\begin{verbatim}
[ 3, "id", "name", "position", 1, "Sánchez", "Manager", 2, "Duffy",
    "Programmer", 3, "Tamburello", "Worker" ]
\end{verbatim}

\subsection{Postup rekonstrukce dat}
\begin{enumerate}
\item Textová data jsou převedena na pole hodnot.
\item Z prvního prvku v poli určíme počet klíčů, označíme $n$.
\item Z prvků 2 až $n+1$ určíme klíče.
\item Z následujících prvků (po $n+1$) rekonstruujeme původní objekty až do konce pole a ukládáme je do pole nového.
\item Nové pole serializujeme jako JSON do řetězce nebo souboru.
\end{enumerate}

\section{CJSON}
\label{cjson}
Proti JSONH dokáže algoritmus CJSON komprimovat i nehomogenní data. K tomu, aby bylo při kompresi dosaženo významné úspory, je nutná určitá struktura dat. Tu bych přirovnal k dědičnosti, jak ji známe z principů objektově orientovaného programování. To znamená, že data lze popsat pomocí tříd, které sdílejí některé členy. Účinnost komprese závisí na poměru počtu tříd a množství dat k nim příslušných. Platí, že čím menší počet tříd a větší množství příslušných dat, tím větší je kompresní účinnost. Za ideál lze potom považovat homogenní data, tedy případ kdy stačí k popisu pouze jedna třída, ke které přísluší všechna data.

Algoritmus v průběhu komprese vytváří postupně strom šablon, který je na závěr vložen do výstupu ve formě jednotlivých šablon, která využívá již zmiňované dědičnosti. Konstrukce stromu je zobrazena na obrázku \ref{cjsonKonstrukceStromu}.

\subsection{Vzorová data}
Mějme kolekci rovinných geometrických útvarů, která obsahuje bod, kružnici a obdélník v tomto pořadí. Tato kolekce může vypadat následujícím způsobem:

\begin{verbatim}
[ { "souřadniceX" : 5, "souřadniceY" : 10 },
    { "souřadniceX" : 8, "souřadniceY" : 4, "poloměr" : 3 },
    { "souřadniceX" : 7, "souřadniceY" : 1, "výška" : 4, "šířka" : 2 } ]
\end{verbatim}

\subsection{Postup komprese}
\label{cjsonPoKompresi}
\begin{enumerate}
\item Textová data jsou převedena na pole objektů.
\item Ze všech objektů v poli jsou postupně vyzvednuty hodnoty a uloženy do nového pole objektů. Tyto objekty obsahují pouze pole příslušných hodnot. Zároveň je konstruován strom šablon.
\item Ze stromu šablon je vytvořeno pole šablon. Šablona je pole obsahující identifikátor šablony, ze které dědí, a klíče. Identifikátor 0 je rezervován pro šablonu odpovídající prázdnému objektu.
\item Objekty s hodnotami jsou doplněny o identifikátor odpovídající šablony.
\item \label{cjsonItem0}Ze vzniklých polí vytvoříme objekt obsahující identifikátor kompresního algoritmu (klíč \texttt{"f"}), pole šablon (klíč \texttt{"t"}) a pole hodnot (klíč \texttt{"v"}) (viz \ref{cjsonPoKompresi}).
\item Objekt vzniklý v bodu \ref{cjsonItem0} serializujeme jako JSON do řetězce nebo souboru.
\end{enumerate}

Během komprese byl sestrojen strom podobně, jako je tomu na obrázku \ref{cjsonKonstrukceStromu}. Popsané uzly (např. bod, kružnice atd.) odpovídají jednotlivým šablonám z nichž šablona pro prázdný objekt je defaultní, se kterou algoritmus začíná a ze které výsledný strom staví.

\begin{figure}[!htb]
\centering
\includegraphics[trim=10 205 210 20, clip, angle=0, width=140mm]{cjsonStrom}
\caption{Postup vytvoření stromu šablon}
\label{cjsonKonstrukceStromu}
\end{figure}

\newpage
Data mají po kompresi následující tvar:

\begin{verbatim}
{   "f" : "cjson",
    "t" : [ [0, "souřadniceX", "souřadniceY"], [1, "poloměr"],
            [1, "výška", "šířka"] ],
    "v" : [ { "" : [1, 5, 10] }, { "" : [2, 8, 4, 3] },
            { "" : [3, 7, 1, 4, 2] } ] }
\end{verbatim}

\subsection{Postup rekonstrukce dat}
\begin{enumerate}
\item Textová data jsou převedena na pole hodnot.
\item Vytvoříme pole, do kterého postupně vkládáme objekty vzniklé z prvků pole hodnot a příslušných šablon.
\item Tuto kolekci serializujeme jako JSON do řetězce nebo souboru.
\end{enumerate}  % Existujících algoritmy pro kompresi JSON
\chapter{Vlastní implementace vybraných kompresních algoritmů}  % Vlastní implementace vybraných kompresních algoritmů
\chapter{Porovnání účinnosti komprese dat ve formátu XML a JSON}  % Porovnání účinnosti komprese dat ve formátech XML a JSON


\chapter*{Závěr} \addcontentsline{toc}{chapter}{Závěr}   % SEM NESAHEJTE!
%V závěru práce byste měli svou práci zhodnotit jako autor. Závěr by tedy měl obsahovat:
%\begin{itemize}
%\item shrnutí výsledků, ke kterým autor dospěl,
%\item přínos autora práce k řešené problematice (co je v práci původní),
%\item zhodnocení využitelnosti dosažených výsledků,
%\item možné pokračování práce (resp. další náměty pro řešení v uvedení oblasti) -- nemusí být řešena Vaší osobou.
%\end{itemize}

%Závěr se většinou vejde na 1 stránku (max. 3 stránky).

Tato závěrečná diplomová práce se věnuje kompresi dat a to hlavně kompresi s využitím znalosti struktury dat. Porovnávány jsou vzájemně datové formáty XML a JSON s cílem určit, který z nich je vhodnější ke kompresi a tím úspornější při archivaci či přenosu dat.

XML je redundantní textový formát, který byl navržen tak, aby byl snadno čitelný pro člověka. Této redundance lze při kompresi XML souboru využít. Formát JSON vznikl jako úsporná alternativa právě ke XML s myšlenkou odstranění redundance dané definicí syntaxe XML. Oba formáty se dnes velice často využívají v prostředí internetu pro práci s daty a proto je vhodné vědět, jaké úspory může jejich komprese přinést.

Praktická část práce je složena ze dvou úkolů. V prvním byla implementována kompresní knihovna, která obsahuje tři algoritmy. Algoritmy Huffmanovo kódování a LZ77 byly zvoleny zejména z toho důvodu, že je úspěšně kombinují algoritmy LZMA2 a gzip, který navíc ke kompresi souborů XML využívá algoritmus XMill. Velmi důležitou součástí implementace kompresního algoritmu je správa paměti, kterou se mi v práci nepodařilo úplně vyřešit.

Dalším praktickým úkolem bylo porovnat účinky komprese dat zapsaných ve formátech XML a JSON pomocí různých kompresních technik. Zvolené algoritmy LZMA2, BZip2, PPMd, gzip, XMILL, JSONH a CJSON byly použity ke komprimaci pěti souborů obsahujících XML a JSON. Tyto soubory byly voleny tak, aby nebyly navzájem podobné velikostí a strukturou dat. Naměřené výsledky jsou v~podobě grafů a tabulky popsány v~kapitole \ref{kapitolaPorovnaniUcinnosti}. Z naměřených dat vyplývá, že moderní kompresní metody, jako jsou LZMA2 a~PPMd, dokáží komprimovat XML i JSON se srovnatelnou nebo dokonce vyšší účinností než algoritmy specializované. Algoritmy JSONH a CJSON dosáhly nejhorších výsledků pro úplně všechny testovací soubory, což je ale dáno naprosto rozdílným způsobem komprimace.

Mezi soubory XML a JSON obsahující stejná data je možné pozorovat významný rozdíl ve velikosti ve prospěch formátu JSON. Toto ale neplatí pro soubory po komprimaci, kdy je rozdíl ve velikosti podstatně menší. Pokud uživatel nevyžaduje možnost práce s~komprimovanými daty, lze doporučit použití klasických běžně dostupných kompresních algoritmů. Na základě výše zmíněných poznatků považuji formát JSON za vhodnější ke kompresi a~práci s daty.  % Závěr, je vkládán z jiného souboru

%  Seznam použitých zdrojů  %%%%%%%%%%%%%%%%%%%%%%
\newpage  % SEM NESAHEJTE!
\renewcommand{\bibname}{Seznam použitých zdrojů} % SEM NESAHEJTE!
\addcontentsline{toc}{chapter}{Seznam použitých zdrojů} % SEM NESAHEJTE!

% !!! Literatura se řadí abecedně.
% !!! Literatura se má seřadit abecedně.
\begin{thebibliography}{99}
\bibitem{json} Standard ECMA-404. {\em The JSON Data Interchange Format.} Geneva: Ecma International, 2013, [online], [cit. 28. října 2014]. Dostupné na: {\tt <http://www.ecma-international.org/publications/files/ECMA-ST/ECMA-404.pdf>}.
\bibitem{idc} The Digital Universe of Opportunities: Rich Data and the Increasing Value of the Internet of Things. INTERNATIONAL DATA CORPORATION. {\em EMC} [online]. 2014 [cit. 2015-02-25]. Dostupné z: {\tt <http://www.emc.com/leadership/digital-universe/2014iview/executive-summary.htm>}.
\bibitem{w3cxml} W3C. {\em Extensible Markup Language (XML) 1.0 (Fifth Edition).} 26. listopadu 2008, [online], [cit. 26. listopadu 2014]. Dostupné na: {\tt <http://www.w3.org/TR/2008/REC-xml-20081126/>}.
\end{thebibliography}
  % text, který je vkládán z jináho souboru, MَŮŽETE ZMĚNIT NÁZEV souboru nebo smazat tento řádek a seznam literatury napsat přímo sem

\newpage % SEM NESAHEJTE!
%  PŘÍLOHY PRÁCE %%%%%%%%%%%%%%%%%%%%%%
\appendix  % přílohy budou opravdu "Přílohy"  :-)     SEM NESAHEJTE!
\addcontentsline{toc}{chapter}{Přílohy} % přidat položku do obsahu      SEM NESAHEJTE!

\part*{Přílohy}  % SEM NESAHEJTE!
\renewcommand{\appendixname}{Příloha} % aby se přílohy nejmenovaly "Dodatek"  SEM NESAHEJTE!

% zde VYMAŽTE NÁZVY vkládaných souborů NEBO sem můžetete rovnou napsat text příloh (a smažte dva následující řádky)

\chapter{Příklady}
\label{prilohaa}

\section{Porovnání XML a JSON}
\label{prilohaPorovnaniXmlJson}
\begin{minipage}{0.5\textwidth}
{\Large XML}\\\\
\texttt{\small
<widget>\\
\hspace*{2mm}<debug>on</debug>\\
\hspace*{2mm}<window\\
\hspace*{4mm}title="Sample Widget\textquotedblright>
\hspace*{4mm}<name>main\_window</name>\\
\hspace*{4mm}<width>500</width>\\
\hspace*{4mm}<height>500</height>\\
\hspace*{2mm}</window>\\
\hspace*{2mm}<image\\
\hspace*{4mm}src="Images/Sun.png"\\
\hspace*{4mm}name="sun1\textquotedblright>\\
\hspace*{4mm}<hOffset>250</hOffset>\\
\hspace*{4mm}<vOffset>250</vOffset>\\
\hspace*{4mm}<alignment>center</alignment>\\
\hspace*{2mm}</image>\\
\hspace*{2mm}<text\\
\hspace*{4mm}data="Click Here"\\
\hspace*{4mm}size="36"\\
\hspace*{4mm}style="bold\textquotedblright>\\
\hspace*{4mm}<name>text1</name>\\
\hspace*{4mm}<hOffset>250</hOffset>\\
\hspace*{4mm}<vOffset>100</vOffset>\\
\hspace*{4mm}<alignment>center</alignment>\\
\hspace*{4mm}<onMouseUp>sun1.opacity =\\
\hspace*{6mm}(sun1.opacity / 100) * 90;\\
\hspace*{4mm}</onMouseUp>\\
\hspace*{2mm}</text>\\
</widget>}
\end{minipage}
\begin{minipage}{0.5\textwidth}
{\Large JSON}\\\\
\texttt{\small
\{ "widget": \{\\
\hspace*{6mm}"debug": "on",\\
\hspace*{6mm}"window": \{\\
\hspace*{8mm}"title": "Sample Widget",\\
\hspace*{8mm}"name": "main\_window",\\
\hspace*{8mm}"width": 500,\\
\hspace*{8mm}"height": 500\\
\hspace*{6mm}\},\\
\hspace*{6mm}"image": \{\\ 
\hspace*{8mm}"src": "Images/Sun.png",\\
\hspace*{8mm}"name": "sun1",\\
\hspace*{8mm}"hOffset": 250,\\
\hspace*{8mm}"vOffset": 250,\\
\hspace*{8mm}"alignment": "center"\\
\hspace*{6mm}\},\\
\hspace*{6mm}"text": \{\\
\hspace*{8mm}"data": "Click Here",\\
\hspace*{8mm}"size": 36,\\
\hspace*{8mm}"style": "bold",\\
\hspace*{8mm}"name": "text1",\\
\hspace*{8mm}"hOffset": 250,\\
\hspace*{8mm}"vOffset": 100,\\
\hspace*{8mm}"alignment": "center",\\
\hspace*{8mm}"onMouseUp": "sun1.opacity =\\
\hspace*{10mm}(sun1.opacity / 100) * 90;"\\\\
\hspace*{6mm}\}\\
\}\}
}
\end{minipage} % příloha 1: vložená z externího souboru
%\chapter{Obsah CD}
\label{prilohab}

Adresáře na CD a jejich obsah:

\begin{description}
\item [Prace] – obsahuje textovou a obrazovou část bakalářské práce včetně všech příloh ve formátu PDF.
\item [Projekt] – obsahuje zdrojové kódy knihovny vytvořené ve vývojovém prostředí MS Visual Studio 2013 a použité externí knihovny.
\item [Knihovna] - obsahuje vytvořenou DLL knihovnu.
\end{description} % příloha 2: vložená z externího souboru

\end{document} % SEM NESAHEJTE!